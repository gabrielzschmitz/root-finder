\section{Introdução}

A busca por zeros de funções é um problema fundamental na matemática aplicada,
engenharia e computação científica. Métodos numéricos desempenham um papel
essencial nessa tarefa, pois muitas funções não possuem soluções analíticas
exatas, tornando necessário o uso de aproximações iterativas.

Neste artigo, exploraremos cinco dos principais algoritmos para encontrar zeros
reais de funções: \textit{Bisseção, Newton-Raphson, Falsa Posição, Ponto Fixo e
Secante}. Cada um desses métodos possui vantagens e limitações, variando em
critérios como taxa de convergência, robustez e eficiência
computacional.\cite{asano2009calculo}

Além de analisar o funcionamento teórico de cada abordagem, desenvolveremos um
programa que permitirá visualizar esses algoritmos em ação. Isso possibilitará
uma compreensão mais intuitiva dos diferentes comportamentos e estratégias
utilizadas para aproximar as raízes de uma função.

Por fim, realizaremos uma comparação detalhada do desempenho de cada método,
avaliando sua eficácia em diferentes cenários. Com essa abordagem, buscamos
oferecer um guia prático e acessível para estudantes, pesquisadores e
profissionais que desejam compreender e aplicar algoritmos de busca de raízes de
forma eficiente.\cite{ruggiero1996calculo}
