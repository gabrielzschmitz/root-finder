\section{Fundamentos Teóricos}

Nesta seção, são apresentados os métodos numéricos implementados. Cada método é
descrito em termos de conceito, funcionamento, convergência e
vantagens/desvantagens.\cite{ruggiero1996calculo}~\cite{asano2009calculo}

\subsection{\textbf{Método da Bisseção}}

O método da bisseção é um algoritmo de busca de raízes que divide repetidamente
um intervalo ao meio e seleciona o subintervalo que contém a
raiz.\cite{bartle1983elementos}~\cite{moreira2011curso}

\subsubsection{Funcionamento}

Dada uma função contínua \( f(x) \) e um intervalo \([a, b]\) onde \( f(a) \cdot
f(b) < 0 \), o método calcula o ponto médio \( c = \frac{a + b}{2} \). Se \(
f(c) = 0 \), \( c \) é a raiz. Caso contrário, o intervalo é reduzido para \([a,
		c]\) ou \([c, b]\), dependendo do sinal de \( f(c) \).

\subsubsection{Convergência}

O método converge linearmente, garantindo uma raiz desde que \( f(x) \) seja
contínua no intervalo e haja uma mudança de sinal. O erro absoluto é reduzido
pela metade a cada iteração, e o número de iterações necessárias para atingir
uma precisão \( \epsilon \) é dado por:
\[
	n \geq \frac{\log(b - a) - \log(\epsilon)}{\log(2)}.
\]

\subsubsection{Vantagens e Desvantagens}

\begin{itemize}
	\item \textbf{Vantagens}: Simplicidade e garantia de convergência.
	\item \textbf{Desvantagens}: Convergência lenta comparada a outros métodos.
\end{itemize}

\subsection{\textbf{Método de Newton-Raphson}}

O método de Newton-Raphson é um algoritmo iterativo que usa a derivada da função
para aproximar a raiz.\cite{moreira2011curso}~\cite{bartle2010introduction}

\subsubsection{Funcionamento}

Dada uma função \( f(x) \) e sua derivada \( f'(x) \), o método começa com uma
estimativa inicial \( x_0 \). A cada iteração, a aproximação é atualizada por:
\[
	x_{n+1} = x_n - \frac{f(x_n)}{f'(x_n)}.
\]
O processo repete até que \( |x_{n+1} - x_n| \) seja menor que uma tolerância
pré-definida.

\subsubsection{Convergência}

O método converge quadraticamente, desde que a estimativa inicial seja próxima
da raiz e \( f'(x) \neq 0 \).

\subsubsection{Vantagens e Desvantagens}

\begin{itemize}
	\item \textbf{Vantagens}: Convergência rápida.
	\item \textbf{Desvantagens}: Requer o cálculo da derivada e pode divergir se
	      a estimativa inicial for inadequada.
\end{itemize}

\subsection{\textbf{Método da Falsa Posição (Regula Falsi)}}

O método da falsa posição é uma variação do método da bisseção que usa uma
aproximação linear para estimar a raiz.\cite{bartle1983elementos}

\subsubsection{Funcionamento}

Dada uma função \( f(x) \) e um intervalo \([a, b]\) onde \( f(a) \cdot f(b) < 0
\), o método calcula a interseção da reta que liga \( (a, f(a)) \) e \( (b,
f(b)) \) com o eixo \( x \):
\[
	c = \frac{a \cdot f(b) - b \cdot f(a)}{f(b) - f(a)}.
\]
Se \( f(c) = 0 \), \( c \) é a raiz. Caso contrário, o intervalo é reduzido para
\([a, c]\) ou \([c, b]\), dependendo do sinal de \( f(c) \).

\subsubsection{Convergência}

Mais rápida que a bisseção, mas ainda linear.

\subsubsection{Vantagens e Desvantagens}

\begin{itemize}
	\item \textbf{Vantagens}: Mais eficiente que a bisseção em muitos casos.
	\item \textbf{Desvantagens}: Pode ser lento se a função for muito
	      não-linear.
\end{itemize}

\subsection{\textbf{Método do Ponto Fixo}}

O método do ponto fixo transforma o problema de encontrar a raiz de \( f(x) = 0
\) em encontrar um ponto fixo \( g(x) = x \).\cite{moreira2011curso}

\subsubsection{Funcionamento}

Reescreva \( f(x) = 0 \) como \( x = g(x) \). Escolha uma estimativa inicial \(
x_0 \) e itere:
\[
	x_{n+1} = g(x_n).
\]
O processo repete até que \( |x_{n+1} - x_n| \) seja menor que uma tolerância
pré-definida.

\subsubsection{Convergência}

Depende da escolha de \( g(x) \). A convergência é garantida se \( |g'(x)| < 1
\) em um intervalo contendo a raiz.

\subsubsection{Vantagens e Desvantagens}

\begin{itemize}
	\item \textbf{Vantagens}: Simplicidade e flexibilidade na escolha de \( g(x) \).
	\item \textbf{Desvantagens}: Pode divergir se \( g(x) \) não for adequada.
\end{itemize}

\subsection{\textbf{Método da Secante}}

O método da secante é uma variação do método de Newton que não requer o cálculo
da derivada.\cite{dequadros2009fundamentos}

\subsubsection{Funcionamento}

Dada uma função \( f(x) \) e duas estimativas iniciais \( x_0 \) e \( x_1 \), o
método calcula a próxima aproximação usando a fórmula:
\[
	x_{n+1} = x_n - f(x_n) \cdot \frac{x_n - x_{n-1}}{f(x_n) - f(x_{n-1})}.
\]
O processo repete até que \( |x_{n+1} - x_n| \) seja menor que uma tolerância
pré-definida.

\subsubsection{Convergência}

Superlinear (ordem \( \approx 1.61\)), mas não tão rápida quanto o método de
Newton.

\subsubsection{Vantagens e Desvantagens}

\begin{itemize}
	\item \textbf{Vantagens}: Não precisa do cálculo da derivada.
	\item \textbf{Desvantagens}: Pode divergir se as estimativas iniciais não forem adequadas.
\end{itemize}

\subsection{\textbf{Comparação Geral dos Métodos}}

A Tabela \ref{tab:comparacao} resume as características principais dos métodos
discutidos. \newline

\begin{table}[H]
	\centering
	\begin{longtable}{|p{1.97cm}|p{1.16cm}|p{1.3cm}|p{1.16cm}|p{1cm}|}
		\hline
		\textbf{} \newline \textbf{Método} & \textbf{Tipo} \newline \textbf{Converg.} & \textbf{Deriv.} \newline \textbf{Necessária?} & \textbf{Garantia} \newline \textbf{Converg.} & \textbf{Veloc.} \newline \textbf{Converg.} \\
		\hline
		\endfirsthead
		\hline
		\endfoot
		\hline
		Bisseção                           & Linear                                   & Não                                           & Sim                                          & Lenta                                      \\
		Newton-Raphson                     & Quadrática                               & Sim                                           & Não                                          & Rápida                                     \\
		Falsa Posição                      & Linear                                   & Não                                           & Sim                                          & Moderada                                   \\
		Ponto Fixo                         & Dep. \( g(x) \)                          & Não                                           & Dep. \( g(x) \)                              & Variável                                   \\
		Secante                            & Superlinear                              & Não                                           & Não                                          & Moderada                                   \\
		\hline
	\end{longtable}
	\caption{Comparação dos métodos numéricos.}
	\label{tab:comparacao}
\end{table}
